\documentclass[letterpaper]{article}
\usepackage[legalpaper, left=1 cm, right=1cm, top=0.5cm, bottom=0.5cm] {geometry}
\date{} % Remove a exibição da data
\usepackage{xcolor}
\usepackage{listings}
\usepackage{graphicx}
\usepackage{hyperref} % Para criar links
\usepackage[utf8]{inputenc}
\usepackage[T1]{fontenc}
\title{\textbf{IMAV}}
\begin{document}
\maketitle
\section{O que é?}
A conferência internacional de micro veiculos "The International Micro Air Vehicles (IMAV)" , é um evento que reune pesquisadores, entusiastas e engenheiros pra discutir sobre os avanços tecnológicos na área de micro veiculos aéreos. Uma das principais atrações deste evento é a combinação de uma competição prática de estudantes com o intercâmbio acadêmico das pesquisas mais recentes em um ambiente de conferência. A conferência conta com a competição de drones não tripulados, competição de detecção de obstaculos , e apresentação de artigos

\section{Onde acontece?}
Bristol -  Reino Unido $-$ 16 a 20 de setembro de 2024
\href{https://2024.imavs.org/}{Site da competição 2024}

\section {Edições passadas}

\begin{itemize}
    \item  O 14º foi realizado em Aachen na Alemanha de 11 de setembro de 2023 a 15 de setembro de 2023 , em DIPOL de HS Nordhausen, Alemanha. em segundo Fantail, Universidade de Auckland, Nova Zelândia, em terceiro Black Bee Drones, Universidade Federal de Itajubá, Brasil. Cabe ressaltar que a equipe black bee , recebeu um premio especial  pela operação MAV , altamente automatizada.
    \href{https://2023.imavs.org/index.php/results-for-imav2023/}{resultados 2023}
    \item O 13º  foi realizado Delft, Holanda, de 12 a 16 de setembro de 2022.
    Na qual a equipe PULP ganhou, e a black bee ficou em segundo lugar, e a skyrats em terceiro
    \href{https://www.imavs.org/2022/index.php/results/index.html}{resultados 2022}
    \item O 12º IMAV 2021 será realizado no México, na cidade de San Andres Cholula, no estado de Puebla, entre os dias 17 e 19 de novembro de 2021.Essa edição por causa da pandemia não teve a competição.
    \href{https://imav2021.inaoep.mx/index.php/portal/home}{resultados 2021}
\end{itemize}
\section{ O que esperar da competição?}
Pra gente, estudantes de graduação competição é a oportunidade de por a prova tudo o que foi desenvolvido ao longo do ano , Tanto pra mecânica quanto pra áreas de programação e marketing, uma competição no Brasil já traz consigo muitas experiencias novas , apesar da ufsc reunir pessoas de diversas partes do Brasil. Porém uma competição no exterior , competindo com pessoas do mundo inteiro tende a trazer consigo um mundo de informações e experiencias que concerteza vai agregar para a equipe quanto para cada um de nós.

\subsection{o que a diretoria espera?}
Pra diretoria vamos ter acesso a áreas que vemos aqui  mas que em uma competição vamos ter um foco maior, como inteligência artificial, Visão computacional, teste e validação de sistemas embarcados. Além de experiencias em gestão e trabalho sobre pressão. lembrando que após uma competição nossos projetos  sempre parecem menores para quem nos tornamos depois  , aprendemos muito que realmente é posto a prova tudo e corrigimos o que faltou.

\section{Missões IMAV}

\subsection{MAVS}
Competição com drones comandados por inteligência artificial
\subsection{obstaculos}




\end{document}
