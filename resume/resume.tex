\documentclass{article}
\usepackage[legalpaper, left=1 cm, right=1cm, top=0.5cm, bottom=0.5cm] {geometry}
\usepackage{hyperref} % Para criar links
\date{}
\title{Orçamento festa junina}
\begin{document}
\maketitle
\tableofcontents
\newpage
\section{competição IMAV 2024}
O documento utilizado para este resumo , buscava descrever , assim como apresentar as regras pormenorizadas do IMAV 2024,  as competições serão apoiadas pela equipa WildDrone e terão um tema de conservação https://wilddrone.eu/
A competição indoor centra-se na operação precisa num ambiente desconhecido. Os principais desafios para esta competição são:
- Operações precisas de MAV em ambientes desconhecidos.
- Interação física do MAV com o ambiente.
- Operações autónomas.
O concurso de exterior centra-se em operações de apoio à conservação utilizando drones num
desconhecido. Os principais desafios para esta competição são:
- Desempenho da aeronave e recolha de informações.
- Planeamento da missão em tempo real.
- Operações autónomas.

\section{Cenário}
A localização exterior em Bristol, Reino Unido, foi recentemente criada como uma reserva de vida selvagem. A reserva está interessada em
utilizar a mais recente tecnologia de drones para monitorizar e recolher informações sobre os animais presentes no local. A sua equipa foi convidada a fazer o seguinte:
- Tarefa 1: Realizar 3 circuitos para os quais será calculada a eficiência do vosso avião.
- Tarefa 2: Efetuar uma busca automática numa determinada área, identificando quantos animais estão
presentes.
- Tarefa 3: A localização de três animais ser-lhe-á dada e deverá automaticamente
capturar automaticamente imagens nítidas de cada animal.
- Tarefa 4: Ser-lhe-á indicado um local alternativo onde a equipa do júri gostaria de
instalar uma armadilha fotográfica. Deve instalar a armadilha fotográfica de forma autónoma. 
\section{Pontuação}
A fórmula de pontuação foi concebida para premiar uma recolha de informações eficiente e totalmente autónoma, sendo a pontuação final global determinada através da seguinte fórmula
a pontuação global final é determinada pela seguinte fórmula:
$$S =\sum_{i = 1}^{n} S_{n}\cdot A_{n} \cdot D_{n}$$
$S_n$: é a pontuação que cada equipa recebe em cada uma das quatro tarefas dadas.\\
$A_n$: está relacionado com o nível de autonomia em que a equipa opera.\\
$D_n$: está ligado à concepção dos MAV utilizados, ou seja, COTS ou estrutura de aeronave de fabrico próprio.\\



\end{document}



This document provides the combined information for both the indoor and outdoor IMAV 2024 competitions. Besides general remarks and a schedule, it gives a detailed description of the competition
areas, the mission elements, and the scoring rules. For IMAV 2024, the competitions will be supported
by the WildDrone team and will have a conservation theme. https://wilddrone.eu/
The indoor competition focuses on precise operation in an unknown environment. The main challenges for this competition are:
• Precise MAV operations in unknown environments.
• MAV physical interaction with the environment.
• Autonomous operations.
The outdoor competition focuses on conservation support operations using drones in an unknown
environment. The main challenges for this competition are:
• Aircraft performance and information-gathering.
• Real-time mission planning.
• Autonomous operations.